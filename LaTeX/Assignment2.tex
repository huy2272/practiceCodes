\documentclass[12pt]{article}
\usepackage[top=1in,bottom=1in,left=1in,right=1in]{geometry}
\usepackage{alltt}
\usepackage{array}	
\usepackage{graphicx}
\usepackage{tabularx}
\usepackage{verbatim}
\usepackage{setspace}
\usepackage{listings}
\usepackage{amssymb,amsmath, amsthm}
\usepackage{hyperref}
\usepackage{oz}
\usepackage[cc]{titlepic}
\usepackage{fancyvrb}


\title{Concordia University\\
Department of Computer Science and Software Engineering\\
\textbf{SOEN 331:\\Formal Methods for Software Engineering}\\
\ \\
\textbf{Exercise in Z}}
\author{}
\date{\today}

\begin{spacing}{1.5}

\begin{document}
\maketitle

\newpage

\section*{Flight monitoring system with the Z specification}

Consider a system such as flightradar24.com. A flight is associated with a flight number
(such as UA79), a specific code that an airline assigns to a particular flight in its network,
and a route which is a source-destination city pair such as (NY , Tokyo). For example, the
United Airlines flight from New to Tokyo is tracked by the system as $ UA79 \mapsto (NY , Tokyo) .$
The formal specification of the system introduces the following three types:

\indent \(FLIGHT\_NUMBER,\)
\newline \indent \(ROUTE,\)
\newline \indent \(CITY \)

\noindent where
\newline \indent $ROUTE: CITY \times CITY$.

\noindent Flight numbers are unique, and there are possibly several flights that cover the same route.
For example, there are possibly several flights from New York to Tokyo. The system must
keep track of all active flights. Formally, let us have the following variables:

\noindent Provide a formal specification in Z, with the following operations:

\begin{enumerate}
	\item \texttt{active}:  holds all active flight numbers.
	\item \texttt{map}: holds a collection of active flight-route pairs.

\end{enumerate}

\newpage

\noindent \underline{Solution}:
\begin{enumerate}
	\item (2 pts) Provide a declaration of variable active.\\
	      $active~:~\mathbb{P}~FLIGHT\_NUMBER$
	\item (3 pts) What kind of collection is variable map?\\
	      \noindent Variable map is a Relation, which is a set of tuples because it holds instances of Cartesian product type ROUTE, and active flight numbers.
	\item (10 pts) Is variable map a function and if so, comment on whether it is a total or
	      partial function, as well as on the properties of injectivity, surjectivity and bijectivity.\\
	      The variable map function is not injective because it is not an one-to-one function.
	      Therefore, the function is not bijective since bijective requires the function to be also injective.
	      The function is surjective because while all flight numbers are unique, they can be assigned to different or to the same route.
	\item (10 pts) Provide a formal specification of the state of the system in terms of a Z specification schema.
	      \begin{class}{flightradar24.com}
		      \begin{schema}{ROUTE}
			      city : CITY\\
				  city2 : CITY\\
			      \where
				  city != city2\\
				  ROUTE: city \times city2
		      \end{schema}\\
		      \begin{schema}{active}
			      active:~\mathbb{P}~FLIGHT\_NUMBER \\
			      \where
			      active \in map\\
		      \end{schema}\\
		      \begin{schema}{map}
				  flight:~\mathbb{P}~FLIGHT\_NUMBER \\
				  map: FLIGHT\_NUMBER \nrightarrow ROUTE\\
				  route: ROUTE 
			      \where
				  flight \in active\\
				  flight = dom\ \ map\\
			      route = ran\ \ map\\
		      \end{schema}\\
	      \end{class}
	      \newpage
	\item (15 pts) Provide a schema for operation RegisterFlightOK that adds a flight to the
	      tracker. With the aid of success and error schema(s), provide a definition for operation
	      RegisterFlight that the system will place in its exposed interface

	      \begin{class}{flightradar24.com}
		      \indent $Message::= ok $\mid already\_active$ $ \\
		      \begin{schema}{Sucess}
			      \Xi RegisterFlight\\
			      response! : Message
			      \where
			      response! = ok
		      \end{schema} \\
		      \begin{schema}{AlreadyActive}
			      \Xi RegisterFlight\\
			      newFlight?: FLIGHT\_NUMBER \\
			      response! = Message
			      \where
			      newFlight \in active \\
			      response! = already\_active
		      \end{schema} \\
		      \begin{schema}{RegisterFlightOK}
			      \Delta (map,active)\\
			      newFlight?: FLIGHT\_NUMBER \\
			      ROUTE? = ROUTE
			      \where
			      newFlight \notin active \\
			      active' = active \cup \{ newFlight? \} \\
			      map' = map \cup \{ newFlight? \mapsto ROUTE? \} \\
		      \end{schema} \\
		      RegisterFlight ~\hat{=}~ (RegisterFlightOK \wedge Success) \oplus (AlreadyActive)
	      \end{class}
	      \newpage
	\item (15 pts) Provide a schema for operation GetRouteOK that returns the route given its
	      flight. With the aid of success and error schema(s), provide a definition for operation
	      GetRoute that the system will place in its exposed interface.
	      \begin{class}{flightradar24.com}
		      \indent $Message::= ok $\mid already\_active \mid$ not\_active $ \\
		      \begin{schema}{Sucess}
			      \Xi Map\\
			      response! : Message
			      \where
			      response! = ok
		      \end{schema} \\
		      \begin{schema}{NotActive}
			      \Xi Map\\
			      flight?: FLIGHT\_NUMBER \\
			      response! = Message
			      \where
			      flight \notin active \\
			      response! = not\_active
		      \end{schema} \\
		      \begin{schema}{GetRouteOK}
			      \Xi Map\\
			      flight?: FLIGHT\_NUMBER \\
			      ROUTE! = ROUTE
			      \where
			      flight \in active \\
			      ROUTE! = map(flight?)
		      \end{schema} \\
		      GetRoute ~\hat{=}~ (GetRouteOK \wedge Success) \oplus (NotActive)
	      \end{class}
	      \newpage
	\item (20 pts) Provide a schema for operation GetFlightOK that returns any and all active
	      flights given a route. With the aid of success and error schema(s), provide a definition
	      for operation GetFlight that the system will place in its exposed interface.
	      \begin{class}{flightradar24.com}
		      \indent $Message::= ok $\mid already\_active \mid not\_active$ $ \\
		      \begin{schema}{Sucess}
			      \Xi Map\\
			      response! : Message
			      \where
			      response! = ok
		      \end{schema} \\
		      \begin{schema}{RouteNotMapped}
			      \Xi Map\\
			      ROUTE?: ROUTE \\
			      flight?: FLIGHT\_NUMBER\\
			      response! = Message
			      \where
			      ROUTE? \notin map(flight?) \\
			      response! = not\_active
		      \end{schema} \\
		      \begin{schema}{GetFlightOK}
			      \Xi Map\\
			      ROUTE? = ROUTE\\
			      active!~:~\mathbb{P}~active \\
			      \where
			      active! = map(ROUTE?)
		      \end{schema} \\
		      GetFlight ~\hat{=}~ (GetFlightOK \wedge Success) \oplus (RouteNotMapped)
	      \end{class}
\end{enumerate}
\end{spacing}
\end{document}
