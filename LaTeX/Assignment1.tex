\documentclass[12pt]{article}
\usepackage{graphics}
\usepackage[top=1in,bottom=1in,left=1in,right=1in]{geometry}
\usepackage{alltt}
\usepackage{array}	
\usepackage{graphicx}
\usepackage{tabularx}
\usepackage{verbatim}
\usepackage{setspace}
\usepackage{listings}

\usepackage{amssymb,amsmath, amsthm}
\usepackage{zed-csp}
\usepackage[cc]{titlepic}
\usepackage{enumitem}
\title{SOEN 331-S:Formal Methods\\for Software Engineering\\
\ \\
Assignment 1}
\author{40125396 Khanh Huy Nguyen, 40178119 Mahanaim Rubin Yo}
\date{October 10, 2022}
\begin{spacing}{1.5}
\begin{document}
\maketitle



\newpage
\section{Predicate Logic}


\noindent In the domain of all people in a room, consider the predicate received request(a, b) that is
interpreted as ``[person] a has received a request from [person] b to connect on some social platform.''
\begin{enumerate}
      \item How are the following two expresions translated into plain English? Are the two expressions logically equivalent?
            \begin{enumerate} [label=\alph*)]
                  \item \( \forall a \exists b \hspace{1mm} received\_request(a,b). \)\\
                        \noindent \underline{Solution}: Every person in the room has received a request to connect on some social media from some (at least one) person.
                  \item \(\exists b \forall a \hspace{1mm} received\_request(a,b).\)\\
                        \noindent \underline{Solution}: There is a person in the room who has requested to connect on some social platfrom to all the people in the room.
                  \item The two expressions a) and b) are {\bf not} logically equivalent.
            \end{enumerate}

      \item Discuss in detail whether we can claim the following:
            \[\forall a \exists b \hspace{1mm} received\_request(a,b) \rightarrow \exists b \forall a \hspace{1mm} received\_request(a,b).\]
            \begin{enumerate} [label=\roman*)]
                  \item If ``Every person in the room has received a request from someone (at least one) in the room to connect on some social platform'',
                        \noindent we cannot safely assumed that ``There is someone who has requested to connect on some social platform to every person in the room''.
                        \noindent It is {\bf not} the case that \(\forall a \exists b \hspace{1mm} received\_request(a,b) \rightarrow \exists b \forall a \hspace{1mm} received\_request(a,b).\)

            \end{enumerate}

      \item Discuss in detail whether we can claim the following:
            \[\exists b \forall a \hspace{1mm} received\_request(a,b) \rightarrow \forall a \exists b \hspace{1mm} received\_request(a,b).\]
            \begin{enumerate} [label=\roman*)]
                  \item If ``There is a person in the room that has requested to connect with everyone in the room on some social platform'',
                        \noindent we can safely assumed that ``Everyone has received a request to connect on some social platform from someone (at least one)''.
                        \noindent It {\bf is} the case that \(\exists b \forall a \hspace{1mm} received\_request(a,b) \rightarrow \forall a \exists b \hspace{1mm} received\_request(a,b).\)
            \end{enumerate}
      \item How are the following two expresions translated into plain English? Are the two expressions logically equivalent?
            \begin{enumerate} [label=\alph*)]
                  \item \( \forall b \exists a \hspace{1mm} received\_request(a,b). \)\\
                        \noindent \underline{Solution}: Everyone in the room has requested to connect on some social platform with someone.
                  \item \(\exists a \forall b \hspace{1mm} received\_request(a,b).\)\\
                        \noindent \underline{Solution}: There is someone in the room whom has received a request to connect on some social platform from everyone in the room.
                  \item The two expressions a) and b) {\bf are} logically equivalent.
            \end{enumerate}

\end{enumerate}


\newpage
\section{Unordered and ordered structures}

\noindent \text Consider the following two sets:
\begin{itemize}
      \item
            \(
            OS =
            \{
            \hspace{1mm} MacOS,
            \hspace{1mm} Linux,
            \hspace{1mm} BSD,
            \hspace{1mm} Windows,
            \hspace{1mm} Unix
            \}, and
            \)
      \item
            \(
            My\_OS =
            \{
            \hspace{1mm} BSD,
            \hspace{1mm} Unix
            \}.
            \)
            \begin{enumerate}
                  \item Is the following declaration acceptable:$My\_OS : \mathbb{P} OS$ ? Explain\\
                        \noindent Yes, because My\_OS is a set, it can assume values supported by $\mathbb{P} OS$. The above declaration is intepreted as ``Variable $My\_OS$ can assume any value supported by $\mathbb{P} OS$''.
                  \item Is $\mathbb{P} OS$ a legitimate type? Explain.\\
                        \noindent Yes, it denotes the type of the elements.
                  \item What is the result of the following statement signify?  $My\_OS : OS.$ Is the statement acceptable? Explain.\\
                        \noindent No, because My\_OS is a set, it cannot assume any value supported by Car.
                  \item Is $ MacOS \in \mathbb{P} OS$ ? Explain.\\
                        \noindent No. MacOs is an atomic variable while it is not a set therefore it cannot be of $\mathbb{P} OS$.
                  \item Is OS a legitimate type?\\
                        \noindent Yes.
                  \item Is $ \{\} \in \mathbb{P} OS$ ? Explain.\\
                        \noindent Yes, the empty set is a subset of any set.
                  \item Is $ \{Linux, BSD\} \in \mathbb{P} OS$ ? Explain.\\
                        \noindent Yes. Variable \{Linux, BSD\} is a set. It is a legitimate element of $\mathbb{P} OS$.
                  \item $ \{\{\}\} \in \mathbb{P} OS$ ? Explain.\\
                        \noindent No, the set containing the empty set is not a subset of $\mathbb{P} OS$
                  \item Is $ \{\} \in OS$ ? Explain.\\
                        \noindent Yes, the empty set is a subset of any set.
                  \item If we define variable $My\_Computer : \mathbb{P} OS$, is \{\} a legitimate value for variable $My\_Computer$? Explain.\\
                        \noindent Yes. The empty set is a subset of any set.
                  \item If we define variable $My\_Computer : \{Windows\}$, would the statement make $My\_Computer$ an atomic variable?\\
                        \noindent Yes.
                  \item Is $\{\{BSD, MacOS\}\} \subset \mathbb{P} OS$ Explain.\\
                        \noindent Yes. The set containing \{BSD, MacOS\} is the subset of the powerset of OS.
                  \item Is $My\_OS \subset \mathbb{P} OS$ Explain.\\
                        \noindent No. $\mathbb{P} OS$ is a set of sets while My\_OS does not contain any sets.
                  \item Is $\{\{BSD, MacOS\}\} \in \mathbb{P} OS$ Explain.\\
                        \noindent No. $\mathbb{P} OS$ does not contain sets that contain another set.
            \end{enumerate}
            \newpage



\end{itemize}
\section{Relational Calculus 2}

\noindent \text{Consider the following binary relation:}
\begin{itemize}
      \item[]
            \( airplanes : Model \leftrightarrow Manufacturer \)

            \noindent where

            \[
                  airplanes = \\
                  \hspace{5mm} \{ \\
                  \hspace{10mm} A320 \mapsto Airbus,\\
                  \hspace{10mm} A330 \mapsto Airbus,\\
                  \hspace{10mm} A350 \mapsto Airbus,\\
                  \hspace{10mm} A380 \mapsto Airbus,\\
                  \hspace{10mm} 737 \mapsto Boeing,\\
                  \hspace{10mm} 747 \mapsto Boeing,\\
                  \hspace{10mm} Superjet100 \mapsto Sukhoi,\\
                  \hspace{10mm} C919 \mapsto Comac,\\
                  \hspace{10mm} Global7500 \mapsto Bombardier,\\
                  \hspace{10mm} Global8000 \mapsto Bombardier,\\
                  \hspace{10mm} E170 \mapsto Embraer\\
                  \hspace{10mm} E175 \mapsto Embraer\\
                  \hspace{5mm} \}
            \]

            \begin{enumerate}

                  \item What is the value of the following expression?\[ \{ A330, 747 \}  \lhd airplanes \]
                        \noindent \underline{Answer}:\\
                        \(\{\\
                        \hspace{10mm} A330 \mapsto Airbus,\\
                        \hspace{10mm} 747 \mapsto Boeing\\
                        \}\)

                  \item What is the result of the expression? \[ airplanes \rhd \{ Comac, Embraer \} \]
                        \noindent \underline{Answer}:\\
                        \(\{\\
                        \hspace{10mm} C919 \mapsto Comac,\\
                        \hspace{10mm} E170 \mapsto Embraer,\\
                        \hspace{10mm} E175 \mapsto Embraer,\\
                        \}\)

                  \item What is the result of the expression? \[\{ A320, A330, A350, E170 \} \ndres airplanes\]
                        \noindent \underline{Answer}:\\
                        \(\{ \\
                        \hspace{10mm} A380 \mapsto Airbus,\\
                        \hspace{10mm} 737 \mapsto Boeing,\\
                        \hspace{10mm} 747 \mapsto Boeing,\\
                        \hspace{10mm} Superjet100 \mapsto Sukhoi,\\
                        \hspace{10mm} C919 \mapsto Comac,\\
                        \hspace{10mm} Global7500 \mapsto Bombardier\\
                        \hspace{10mm} Global8000 \mapsto Bombardier\\
                        \hspace{10mm} E175 \mapsto Embraer\\
                        \}\)
                  \item What is the result of the expression? \[ airplanes \nrres \{ Airbus, Boeing \} \]
                        \noindent \underline{Answer}:\\
                        \(\{ \\
                        \hspace{10mm} Superjet100 \mapsto Sukhoi,\\
                        \hspace{10mm} C919 \mapsto Comac,\\
                        \hspace{10mm} Global7500 \mapsto Bombardier,\\
                        \hspace{10mm} Global8000 \mapsto Bombardier,\\
                        \hspace{10mm} E170 \mapsto Embraer,\\
                        \hspace{10mm} E175 \mapsto Embraer\\
                        \}\)
                  \item Consider the following expression: \[ airplanes \oplus \{ Su\_80 \mapsto Sukhoi \} \]
                        \noindent Comment on the deployment of the expression in the context of a database table.
                        \noindent \underline{Answer}:\\
                        \(
                        airplanes = \\
                        \hspace{5mm} \{ \\
                        \hspace{10mm} Su\_80 \mapsto SuKhoi,\\
                        \hspace{10mm} A320 \mapsto Airbus,\\
                        \hspace{10mm} A330 \mapsto Airbus,\\
                        \hspace{10mm} A350 \mapsto Airbus,\\
                        \hspace{10mm} A380 \mapsto Airbus,\\
                        \hspace{10mm} 737 \mapsto Boeing,\\
                        \hspace{10mm} 747 \mapsto Boeing,\\
                        \hspace{10mm} Superjet100 \mapsto Sukhoi,\\
                        \hspace{10mm} C919 \mapsto Comac,\\
                        \hspace{10mm} Global7500 \mapsto Bombardier,\\
                        \hspace{10mm} Global8000 \mapsto Bombardier,\\
                        \hspace{10mm} E170 \mapsto Embraer\\
                        \hspace{10mm} E175 \mapsto Embraer\\
                        \hspace{5mm} \}\)\\
                        \noindent Relational overriding can model database updates (addition of records).
                        However, the expression does not have apermanent effect on the database since it is not an assignment statement.
            \end{enumerate}
\end{itemize}
\end{spacing}
\end{document}
