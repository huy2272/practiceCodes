\documentclass[12pt]{article}
\usepackage{graphics}
\usepackage[top=1in,bottom=1in,left=1in,right=1in]{geometry}
\usepackage{alltt}
\usepackage{array}	
\usepackage{graphicx}
\usepackage{tabularx}
\usepackage{verbatim}
\usepackage{setspace}
\usepackage{listings}

\usepackage{amssymb,amsmath, amsthm}
\usepackage{zed-csp}
\usepackage[cc]{titlepic}
\usepackage{enumitem}
\title{SOEN 331-S:Formal Methods\\for Software Engineering\\
\ \\
Assignment 1}
\author{Khanh Huy Nguyen, Mahanaim Rubin Yo}
\date{\today}
\begin{spacing}{1.5}
\begin{document}
\maketitle



\newpage
\section{Predicate Logic}


\noindent In the domain of all people in a room, consider the predicate received request(a, b) that is
interpreted as ``[person] a has received a request from [person] b to connect on some social platform.''
\begin{enumerate}
      \item How are the following two expresions translated into plain English? Are the two expressions logically equivalent?
            \begin{enumerate} [label=\alph*)]
                  \item \( \forall a \exists b \hspace{1mm} received\_request(a,b). \)\\
                        \noindent \underline{Solution}: Every person in the room has received a request to connect on some social media from some (at least one) person.
                  \item \(\exists b \forall a \hspace{1mm} received\_request(a,b).\)\\
                        \noindent \underline{Solution}: There is a person in the room who has requested to connect on some social platfrom to all the people in the room.
                  \item The two expressions a) and b) are {\bf not} logically equivalent.
            \end{enumerate}

      \item Discuss in detail whether we can claim the following:
            \[\forall a \exists b \hspace{1mm} received\_request(a,b) \rightarrow \exists b \forall a \hspace{1mm} received\_request(a,b).\]
            \begin{enumerate} [label=\roman*)]
                  \item If ``Every person in the room has received a request from someone (at least one) in the room to connect on some social platform'',
                        \noindent we cannot safely assumed that ``There is someone who has requested to connect on some social platform to every person in the room''.
                        \noindent It is {\bf not} the case that \(\forall a \exists b \hspace{1mm} received\_request(a,b) \rightarrow \exists b \forall a \hspace{1mm} received\_request(a,b).\)

            \end{enumerate}

      \item Discuss in detail whether we can claim the following:
            \[\exists b \forall a \hspace{1mm} received\_request(a,b) \rightarrow \forall a \exists b \hspace{1mm} received\_request(a,b).\]
            \begin{enumerate} [label=\roman*)]
                  \item If ``There is a person in the room that has requested to connect with everyone in the room on some social platform'',
                        \noindent we can safely assumed that ``Everyone has received a request to connect on some social platform from someone (at least one)''.
                        \noindent It {\bf is} the case that \(\exists b \forall a \hspace{1mm} received\_request(a,b) \rightarrow \forall a \exists b \hspace{1mm} received\_request(a,b).\)
            \end{enumerate}
      \item How are the following two expresions translated into plain English? Are the two expressions logically equivalent?
            \begin{enumerate} [label=\alph*)]
                  \item \( \forall b \exists a \hspace{1mm} received\_request(a,b). \)\\
                        \noindent \underline{Solution}: Everyone in the room has requested to connect on some social platform with someone.
                  \item \(\exists a \forall b \hspace{1mm} received\_request(a,b).\)\\
                        \noindent \underline{Solution}: There is someone in the room whom has received a request to connect on some social platform from everyone in the room.
                  \item The two expressions a) and b) {\bf are} logically equivalent.
            \end{enumerate}

\end{enumerate}


\newpage
\section{Unordered and ordered structures}

\noindent \text Consider the following two sets:
\begin{itemize}
      \item
            \(
            OS =
            \{
            \hspace{1mm} MacOS,
            \hspace{1mm} Linux,
            \hspace{1mm} BSD,
            \hspace{1mm} Windows,
            \hspace{1mm} Unix
            \}, and
            \)
      \item
            \(
            My\_OS =
            \{
            \hspace{1mm} BSD,
            \hspace{1mm} Unix
            \}.
            \)
            \begin{enumerate}
                  \item Is the following declaration acceptable:$My\_OS : \mathbb{P} OS$ ? Explain

                  \item Is $\mathbb{P} OS$ a legitimate type? Explain

                  \item What is the result of the following statement signify?  $My\_OS : OS.$ Is the statement acceptable? Explain

                  \item Is $ MacOS \in \mathbb{P} OS$ ? Explain

                  \item Is OS a legitimate type?
                   
                  \item Is $ \{\} \in \mathbb{P} OS$ ? Explain

                  \item Is $ \{Linux, BSD\} \in \mathbb{P} OS$ ? Explain
                        
                  \item $ \{\{\}\} \in \mathbb{P} OS$ ? Explain
                  
                  \item Is $ \{\} \in OS$ ? Explain
                  
                  \item If we define variable $My\_Computer : \mathbb{P} OS$, is \{\} a legitimate value for variable $My\_Computer$? Explain.
                  
                  \item If we define variable $My\_Computer : \{Windows\}$, would the statement make $My\_Computer$ an atomic variable?
                  
                  \item Is $\{\{BSD, MacOS\}\} \subset \mathbb{P} OS$ Explain.
                  
                  \item Is $My\_OS \subset \mathbb{P} OS$ Explain.
                  
                  \item Is $\{\{BSD, MacOS\}\} \in \mathbb{P} OS$ Explain.
            \end{enumerate}
            \newpage
            \begin{enumerate}

                  \item How do we interpret the expression $Favorite : \mathbb{P} Phone$?

                  \item Is $\mathbb{P} Phone$ a legitimate type?

                  \item What is the nature of the variable in $Favorite : \mathbb{P} Phone$? (i.e. atomic or composite?)

                  \item Is $Apple \in \mathbb{P} Phone$?

                  \item Is $\{ Apple \} \in \mathbb{P} Phone$?

                  \item Is $\{ \{ \} \} \in \mathbb{P} Phone$?

                  \item Is $\{ \} \in \mathbb{P} Phone$?

                  \item If we define variable $Favorite : \mathbb{P} Phone$, is $\{ \}$ a legitimate value for variable $Favorite$?

                  \item Is $Favorite \in \mathbb{P} Phone$?

                  \item Is $Favorite \subset \mathbb{P} Phone$?

            \end{enumerate}

            \noindent \underline{Solution}:

            \begin{enumerate}

                  \item How do we interpret the expression $Favorite : \mathbb{P} Phone$? \underline{Answer}: This is  interpreted as ``The variable $Favorite$ can assume any value supported by the powerset of $Phone$.

                  \item Is $\mathbb{P} Phone$ a legitimate type? \textbf{Yes.}

                  \item What is the nature of the variable in $Favorite : \mathbb{P} Phone$? \underline{Answer}: Variable $Favorite$ is a \textbf{set}.

                  \item Is $Apple \in \mathbb{P} Phone$? \textbf{No}.

                  \item Is $\{ Apple \} \in \mathbb{P} Phone$?  \textbf{Yes}.

                  \item Is $\{ \{ \} \} \in \mathbb{P} Phone$? \textbf{No}.

                  \item Is $\{ \} \in \mathbb{P} Phone$? \textbf{Yes}.

                  \item If we define variable $favorite : \mathbb{P} Phone$, is $\{ \}$ a legitimate value for variable $Favorite$? \textbf{Yes}.

                  \item Is $Favorite \in \mathbb{P} Phone$? \textbf{Yes}.

                  \item Is $Favorite \subset \mathbb{P} Phone$? \textbf{No}.

            \end{enumerate}

      \item Consider the following relation:

            \[ laptops : Model \leftrightarrow Brand \]

            \noindent where

            \[
                  laptops = \\
                  \hspace{5mm} \{ \\
                  \hspace{10mm} legion5 \mapsto lenovo,\\
                  \hspace{10mm} macbookair \mapsto apple,\\
                  \hspace{10mm} xps15 \mapsto dell,\\
                  \hspace{10mm} spectre \mapsto hp,\\
                  \hspace{10mm} xps13 \mapsto dell,\\
                  \hspace{10mm} swift3 \mapsto acer,\\
                  \hspace{10mm} macbookpro \mapsto apple,\\
                  \hspace{10mm} dragonfly \mapsto hp,\\
                  \hspace{10mm} envyx360 \mapsto hp\\
                  \hspace{5mm} \}
            \]

            \begin{enumerate}

                  \item What is the domain and the range of the relation?

                        \noindent \underline{Answer}:

                        \begin{itemize}
                              \item The domain is defined as:  % \textbf{\underline{Marking:  0.5 pt}}
                                    \[
                                          \dom laptops = \\
                                          \hspace{5mm} \{ \\
                                          \hspace{10mm} legion5,\\
                                          \hspace{10mm} macbookair,\\
                                          \hspace{10mm} xps15,\\
                                          \hspace{10mm} spectre,\\
                                          \hspace{10mm} xps13,\\
                                          \hspace{10mm} swift3,\\
                                          \hspace{10mm} macbookpro,\\
                                          \hspace{10mm} dragonfly,\\
                                          \hspace{10mm} envyx360\\
                                          \hspace{5mm} \}
                                    \]


                              \item The range is defined as:  $\ran~laptops = \{ lenovo, apple, dell, hp, acer \}$.


                        \end{itemize}


                  \item What is the result of the expression

                        \[ \{ xps15, xps13, swift3, envyx360 \}  \lhd laptops \]

                        \noindent What is the meaning of operator $\lhd$ and where would you deploy such operator in the context of a database management system?

                        \newpage

                        \noindent \underline{Answer}:

                        \noindent The result is

                        \[ \{ xps15, xps13, swift3, envyx360 \}  \lhd laptops = \\
                              \hspace{5mm} \{ \\
                              \hspace{10mm} xps15 \mapsto dell,\\
                              \hspace{10mm} xps13 \mapsto dell,\\
                              \hspace{10mm} swift3 \mapsto acer,\\
                              \hspace{10mm} envyx360 \mapsto hp\\
                              \hspace{5mm} \}
                        \]

                        \noindent Domain restriction selects pairs based on their first element. We deploy such operators to model database queries.


                  \item What is the result of the expression

                        \[ laptops \rhd \{ lenovo, hp \} \]

                        \noindent What is the meaning of operator $\rhd$ and where would you deploy such operator in the context of a database management system?

                        \noindent \underline{Answer}:

                        \noindent The result is


                        \[ laptops \rhd \{ lenovo, hp \} = \\
                              \hspace{5mm} \{ \\
                              \hspace{10mm} legion5 \mapsto lenovo,\\
                              \hspace{10mm} spectre \mapsto hp,\\
                              \hspace{10mm} dragonfly \mapsto hp,\\
                              \hspace{10mm} envyx360 \mapsto hp\\
                              \hspace{5mm} \}
                        \]

                        \noindent Range restriction selects pairs based on their second element. We deploy such operators to model database queries.


                  \item What is the result of the expression

                        \[ \{ legion5, xps15, xps13, dragonfly \} \ndres laptops \]

                        \noindent What is the meaning of operator $\ndres$ and where would you deploy such operator in the context of a database management system?

                        \noindent \underline{Answer}:

                        \noindent \noindent The result is

                        \[ \{ legion5, xps15, xps13, dragonfly \} \ndres laptops = \\
                              \hspace{5mm} \{ \\
                              \hspace{10mm} macbookair \mapsto apple,\\
                              \hspace{10mm} spectre \mapsto hp,\\
                              \hspace{10mm} swift3 \mapsto acer,\\
                              \hspace{10mm} macbookpro \mapsto apple,\\
                              \hspace{10mm} envyx360 \mapsto hp\\
                              \hspace{5mm} \}
                        \]

                        \noindent  Domain subtraction removes elements from the domain of the relation. We deploy such operation to model deletion of records.


                  \item What is the result of the expression

                        \[ laptops \nrres \{ apple, dell, hp \} \]

                        \noindent What is the meaning of operator $\nrres$ and where would you deploy such operator in the context of a database management system?

                        \newpage

                        \noindent \underline{Answer}:


                        \noindent \noindent The result is

                        \[
                              laptops \nrres \{ apple, dell, hp \} = \\
                              \hspace{5mm} \{ \\
                              \hspace{10mm} legion5 \mapsto lenovo,\\
                              \hspace{10mm} swift3 \mapsto acer\\
                              \hspace{5mm} \}
                        \]


                        \noindent  Range subtraction removes elements from the codomain of the relation. We deploy such operation to model database updates (deletion of records). % \textbf{\underline{Marking:  1.5 pts}}



                  \item Consider the following expression

                        \[ laptops \oplus \{ ideapad \mapsto lenovo \} \]


                        \begin{enumerate}
                              \item What is the result of the expression?

                              \item What is the meaning of operator $\oplus$ and where would you deploy such operator in the context of a database management system?

                              \item Does the result of the expression have a permanent effect on the database (relation)? If not, describe in detail how would you ensure a permanent effect.

                        \end{enumerate}

                        \newpage

                        \noindent \underline{Answer}:


                        \begin{enumerate}
                              \item  The result is

                                    \[ laptops \oplus \{ ideapad \mapsto lenovo \} = \\
                                          \hspace{5mm} \{ \\
                                          \hspace{10mm} ideapad \mapsto lenovo,\\
                                          \hspace{10mm} legion5 \mapsto lenovo,\\
                                          \hspace{10mm} macbookair \mapsto apple,\\
                                          \hspace{10mm} xps15 \mapsto dell,\\
                                          \hspace{10mm} spectre \mapsto hp,\\
                                          \hspace{10mm} xps13 \mapsto dell,\\
                                          \hspace{10mm} swift3 \mapsto acer,\\
                                          \hspace{10mm} macbookpro \mapsto apple,\\
                                          \hspace{10mm} dragonfly \mapsto hp,\\
                                          \hspace{10mm} envyx360 \mapsto hp\\
                                          \hspace{5mm} \}
                                    \]

                              \item Relational overriding can model database updates (addition of records).


                              \item The expression does not have a permanent effect on the database (relation). To ensure a permanent effect on the relation, we need to define an assignment statement

                                    \[ laptops' = laptops \oplus \{ ideapad \mapsto lenovo \} \]

                                    \noindent which reads ``The value of variable (relation) $laptops$ is assigned the result of the expression on the right-hand-side of the assignment statement.''

                        \end{enumerate}
            \end{enumerate}

\end{itemize}

\end{spacing}

\end{document}
